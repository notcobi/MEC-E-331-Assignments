% Formula sheet for MEC E 331 Fluid mechanics 1
% Two column format

\documentclass[10pt]{article}
\usepackage{amsmath}
\usepackage{amssymb}
\usepackage{multicol}
\usepackage{geometry}
\usepackage{fancyhdr}
\usepackage{siunitx}
\usepackage{enumitem}
\usepackage{multicol}
\usepackage{graphicx}
\usepackage{multirow}
\usepackage{lastpage}
\usepackage[final]{hyperref}
\usepackage{parskip}

\DeclareRobustCommand{\volume}{\text{\volumedash}V}
\newcommand{\volumedash}{%
  \makebox[0pt][l]{%
    \ooalign{\hfil\hphantom{$\m@th V$}\hfil\cr\kern0.08em--\hfil\cr}%
  }%
}

\geometry{letterpaper, portrait, margin=0.5in, footskip=0.25in, top = 0.75in, headsep=0.25in}
\setlength{\columnsep}{0.5in}

\hypersetup{
	colorlinks=true,       % false: boxed links; true: colored links
	linkcolor=blue,        % color of internal links
	citecolor=blue,        % color of links to bibliography
	filecolor=magenta,     % color of file links
	urlcolor=blue         
}

\pagestyle{fancy}
\fancyhf{}
\lhead{MEC E 371 Formula Sheet}
\chead{Last Updated: \today}
\rhead{Alex Diep}
\cfoot{\thepage\ of \pageref{LastPage}}
%add line for footer 
% \renewcommand{\footrulewidth}{0.4pt}% default is 0pt

% set default font to sans-serif
\renewcommand{\familydefault}{\sfdefault}

% set head width = 18cm
\setlength{\headwidth}{18cm}

% set head height and top = 1.2cm
\setlength{\headheight}{15pt}

\begin{document}
%\maketitle
%\thispagestyle{empty}
\begin{multicols*}{2}
\section*{9. Differential Analysis of Fluid Flow}
\subsection*{8.1. General Procedure}
\begin{enumerate}
    \item a
\end{enumerate}

% Add this to a glossary section later
\subsection*{8.2. Variable Definitions}
\begin{itemize}
    \item $\nabla$: The gradient operator, $\nabla = \hat{i} \frac{\partial}{\partial x} + \hat{j} \frac{\partial}{\partial y} + \hat{k} \frac{\partial}{\partial z}$
    \item $\frac{D}{Dt}$: The material derivative, $\frac{D}{Dt} = \frac{\partial}{\partial t} + \vec{V} \cdot \nabla$
    \item $\vec{V}$: Velocity vector, $\vec{V} = u \hat{i} + v \hat{j} + w \hat{k}$
    \item $\nabla^2$: The Laplacian operator
    \item $P$: Pressure
\end{itemize}

\subsection*{8.3. Formulas}
\vspace{-0.4cm}
\begin{align*}
    &\text{Continuity Equation: } \frac{\partial \rho}{\partial t} + \nabla \cdot (\rho \vec{V}) = 0 \\
    &\text{Cartesian: } \frac{\partial \rho}{\partial t} + \frac{\partial}{\partial x}(\rho u) + \frac{\partial}{\partial y}(\rho v) 
    + \frac{\partial}{\partial z}(\rho w) = 0 \\
    &\text{Cylindrical: } \frac{\partial \rho}{\partial t} + \frac{1}{r} \frac{\partial(r \rho u)}{\partial r} + \frac{\partial(\rho v)}{\partial \theta} 
    + \frac{\partial(\rho w)}{\partial z} = 0 \\
    &\text{Special Case 1: Steady Compressible Flow: } \nabla \cdot (\rho \vec{V}) = 0 \\
    &\text{Special Case 2: Incompressible Flow: } \nabla \cdot \vec{V} = 0 
\end{align*}
Incompressible flow, constant $\mu$, Navier-Stokes Equation:
\vspace{-0.1cm}
\begin{align*}
    \rho \frac{ D \vec{V}}{D t}  &= -\nabla P + \rho \vec{g} + \mu \nabla^2 \vec{V} \\
\end{align*}
\vspace{-0.5cm}
Constant $\dot{q}$:
\begin{align*}
    T_e &= T_i + \frac{\dot{q}}{\dot{m} c_p} \\ 
    \dot{q} = h(T_s-T_b)
\end{align*}
Constant $T_s$:
\vspace{-0.5cm}
\begin{align*}
    T_e &= T_s - (T_s - T_i) \exp\left(-\frac{\dot{m} C_p}{h A_s}\right) \\
    T_s &=\frac{T_e - T_i \exp\left(-\frac{\dot{m} C_p}{h A_s}\right)}{1 - \exp\left(-\frac{\dot{m} C_p}{h A_s}\right)} \\
    \dot{Q} &= h A_s \Delta T_{\text{lm}} \\
    T_{\text{lm}} &= \frac{T_{i} - T_{e}}{\ln[(T_{s} - T_{e})/(T_{s} - T_{i})]} 
\end{align*}    
For fully developed laminar flow, use Table \ref{tab:sec8_fully_developed_laminar}.

For entry region in a circular tube where $T_s = \text{constant}$, use:
\vspace{-0.5cm}
\begin{align*}
    \text{(Edwards et al., 1979) } \text{Nu} = 3.66 + \frac{0.0658(D/L) \text{Re} \text{Pr}}{1 + 0.04[(D/L) \text{Re} \text{Pr}]^{2/3}} 
\end{align*}

For entry region in a circular tube where the difference between $T_s$ and $T_b$ is large, use:
\vspace{-0.3cm}
\begin{align*}
    \text{(Sieder and Tate, 1936) } &\text{Nu} = 1.86\left(\frac{\text{Re} \text{Pr} D}{L}\right)^{1/3} \left(\frac{\mu_b}{\mu_s}\right)^{0.14} \\
    0.6 < \text{Pr} < 5, &\quad 0.0044 < \frac{\mu_b}{\mu_s} < 9.75
\end{align*}
All properties for Sieder and Tate should be evaluated at $T_b$ except $\mu_s$ which should be evaluated at $T_s$.

For entry region between two isothermal parallel plates, use:
\vspace{-0.3cm}
\begin{align*}
    \text{(Edwards et al., 1979) } &\text{Nu} = 7.54 + \frac{0.03(D_h/L) \text{Re} \text{Pr}}{1 + 0.016[(D_h/L) \text{Re} \text{Pr}]^{2/3}} \\
    &\text{Re} \leq 2800
\end{align*}
For turbulent flow in a circular tube, use:
\vspace{-0.3cm}
\begin{align*}
    \text{(Dittus-Boelter, 1930) } \text{Nu} = 0.023 \text{Re}^{0.8} \text{Pr}^{n}  \\
    n = 0.4 \; (\text{Heating}), \quad n = 0.3 \; (\text{Cooling}) 
\end{align*}

\subsection*{8.4 General Terms}
\begin{itemize}
    \item Control volume analysis: A method of analysis in which a volume in space is selected and the conservation of mass, momentum, and energy are applied to the volume
    \item Differential analysis: involves application of differential equations of fluid motion to any and every point in the flow field over a region called the flow domain.
\end{itemize}

\end{multicols*}
\end{document}
