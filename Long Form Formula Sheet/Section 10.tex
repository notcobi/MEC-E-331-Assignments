\section*{10. Boundary Layer Approximation}
\subsection*{10.1 General Procedure}
\begin{enumerate}
    \item Identify the type of flow using the Reynolds number. If the Re $> 5 \times 10^5$, the flow is turbulent. If the Re $< 5 \times 10^5$, the flow is laminar.
    \item Use Table \ref{tab:boundary_layer_approximation} to determine whatever you need.
\end{enumerate}

\subsection*{10.2 Variable Definitions}
\begin{itemize}
    \item Re$_x$ = Reynolds number, the ratio of inertial forces to viscous forces, at $x$
    \item $\delta$ = Boundary layer thickness is the distance from the wall to the point where the velocity is 99\% of the free stream velocity.
    \item $\delta*$ = Displacement thickness is the distance that a streamline just outside 
    of the boundary layer is deflected away from the wall due to the effect 
    of the boundary layer.
    \item $\theta$ = Momentum thickness, defined as the loss of momentum flux per unit width decided by $\rho U^2$
    due to the presence of the growing boundary layer.
    \item $\tau_w$ = Wall shear stress, the force per unit area exerted by the fluid on the wall.
    \item $C_f$ = Local friction coefficient, the ratio of the wall shear stress to the dynamic pressure.
    \item 
\end{itemize}

\subsection*{10.3 Formulas}
\begin{fleqn}
    \begin{align*}
        &\text{Re}_x = \frac{\rho V x}{\mu}  = \frac{V x}{\nu} \\
        &\text{Boundary Layer Thickness: } \frac{\delta}{x} = 4.91 \sqrt{\text{Re}_x} \\
        &\text{Wall Shear Stress: } \tau_w = \frac{0.332 \rho U^2}{\sqrt{\text{Re}_x}} \\
        &\text{Local Friction Coefficient: } C_f = \frac{\tau_w}{\frac{1}{2} \rho U^2} = \frac{0.664}{\sqrt{\text{Re}_x}} \\
        &\text{Displacement Thickness: } \delta^* = \int_0^\infty \left(1 - \frac{u}{U}\right) dy = \frac{1.72 x}{\sqrt{\text{Re}_x}} \\
        &\text{Momentum Thickness: } \theta = \int_0^\infty \frac{u}{U} \left(1 - \frac{u}{U}\right) dy = \frac{0.664 x}{\sqrt{\text{Re}_x}} \\   
        &\text{Drag Force: } F_D = \int_A \tau_w dA = \int_0^L \tau_w w dx \\
        &\text{Continuity Equation: } \frac{\partial u}{\partial x} + \frac{\partial v}{\partial y} = 0 \\
        &\text{Momentum Equation: } u \frac{\partial u}{\partial x} + v \frac{\partial u}{\partial y} = U \frac{dU}{dx} + \nu \frac{\partial^2 u}{\partial y^2} \\
    \end{align*}
\end{fleqn}

\begin{table}[H]
    \caption{Boundary Layer Approximation for flat plate.}
    \label{tab:boundary_layer_approximation}
    \centering
    \begin{tabular}{cc}
        \hline
        Laminar & Turbulent \\
        \hline
        $\frac{\delta}{x} = 4.91 \sqrt{\text{Re}_x}$ & $\frac{\delta}{x} = \frac{0.16}{(\text{Re}_x)^{1/7}}$ \\
        $\frac{\delta^*}{x} = \frac{1.72}{\sqrt{\text{Re}_x}}$ & $\frac{\delta^*}{x} = \frac{0.020}{(\text{Re}_x)^{1/7}}$ \\
        $\frac{\theta}{x} = \frac{0.664}{\sqrt{\text{Re}_x}}$ & $\frac{\theta}{x} = \frac{0.016}{(\text{Re}_x)^{1/7}}$ \\   
        $C_f = \frac{0.664}{\sqrt{\text{Re}_x}}$ & $C_f = \frac{0.027}{(\text{Re}_x)^{1/7}}$ \\
        $\tau_w = \frac{0.332 \rho U^2}{\sqrt{\text{Re}_x}}$ & $\tau_w = \frac{0.013 \rho U^2}{(\text{Re}_x)^{1/7}}$ \\
        \hline
    \end{tabular}
\end{table}