\section*{Terms}
% Static Pressure: the actual thermodynamic pressure dynamic Pressure: represents the pressure rise when fluid motion is brought to a stop isentropically Hydrostatic pressure: value depends on the reference level selected where z=0. A ‘potential’ pressure. Total Pressure = static pressure + dynamic pressure hydrostatic pressure. Stagnation pressure= static pressure+dynamic pressure. Pressure head: height of fluid needed to produce a static pressure P velocity head: height of fluid needed to produce velocity V during a free, frictionless vertical fall. Elevation head: height relative to some reference plate z=0. Hydraulic Grade Line(HGL): the line that represents pressure and elevation heads.Energy Grade Line (EGL): the line that represents the total head. Obstruction flow meters: obstruction flow meters intentionally constrict the flow, and measure the drop in static pressure at the constriction. This pressure drop is proportional to the flow velocity through the constriction. Control volume analysis: very useful tool to engineering for flow analysis. Gives ‘engineering analysis’ answer, sometimes crude approximation, but always useful. Differential (small-scale) analysis: in principle can be used for any problems. In practice, limited cases where exact analytical solutions exist. Nowadays, CFD (computational fluid dynamics) simulations are widely performed based on differential analysis. Experimental (dimensional) analysis: based on the results of experiments. Technique to derive the most use out of the fewest number of experiments (which cost time and money). Displacement thickness: the distance that a streamline just outside the boundary layer is deflected away from the wall due to the effect of the boundary layer. The imaginary increase in the thickness of the wall, as seen by the outer flow due to the effect of the growing boundary layer. Momentum thickness: a measure of the loss of momentum flux per unit width. Thickness of a 
% layer of fluid velocity U for which the momentum flux is equal to the deficit of momentum flux through the boundary layer. 


%Momentum thickness is defined as the loss of momentum flux per unit width divided by 𝝆U^2 due to the presence of the growing boundary layer. 𝝆U^2𝜃 the loss of momentum flux per unit width due to the growing boundary layer.
\begin{itemize}
    \item Static Pressure: the actual thermodynamic pressure
    \item Dynamic Pressure: represents the pressure rise when fluid motion is brought to a stop isentropically
    \item Hydrostatic pressure: value depends on the reference level selected where z=0. A 'potential' pressure.
    \item Total Pressure = static pressure + dynamic pressure hydrostatic pressure.
    \item Stagnation pressure= static pressure+dynamic pressure.
    \item Pressure head: height of fluid needed to produce a static pressure P
    \item Velocity head: height of fluid needed to produce velocity V during a free, frictionless vertical fall.
    \item Elevation head: height relative to some reference plate z=0.
    \item Hydraulic Grade Line(HGL): the line that represents pressure and elevation heads.
    \item Energy Grade Line (EGL): the line that represents the total head.
    \item Obstruction flow meters: obstruction flow meters intentionally constrict the flow, and measure the drop in static pressure at the constriction. This pressure drop is proportional to the flow velocity through the constriction.
    \item Control volume analysis: very useful tool to engineering for flow analysis. Gives 'engineering analysis' answer, sometimes crude approximation, but always useful.
    \item Differential (small-scale) analysis: in principle can be used for any problems. In practice, limited cases where exact analytical solutions exist. Nowadays, CFD (computational fluid dynamics) simulations are widely performed based on differential analysis.
    \item Experimental (dimensional) analysis: based on the results of experiments. Technique to derive the most use out of the fewest number of experiments (which cost time and money).
    \item Displacement thickness: the distance that a streamline just outside the boundary layer is deflected away from the wall due to the effect of the boundary layer. The imaginary increase in the thickness of the wall, as seen by the outer flow due to the effect of the growing boundary layer
    \item Momentum thickness: defined as the loss of momentum flux per unit width divided by $\rho U^2$ due to the presence of the growing boundary layer. $\rho U^2 \theta$ the loss of momentum flux per unit width due to the growing boundary layer.
\end{itemize}
